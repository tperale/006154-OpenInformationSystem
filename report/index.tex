\documentclass[a4paper,11pt]{article}

\usepackage[english]{babel}
\usepackage[utf8x]{inputenc}
\usepackage{amsmath}
\usepackage{graphicx}
\usepackage{cite}
\usepackage{hyperref}
\usepackage{fullpage}
% \usepackage{apacite}
\setlength{\parskip}{1.2ex}
\begin{document}

\title{Open Information System: Conceptual schema}
\author{Thomas Perale -- 0546990\\Maximilien Romain -- 0543411\\Felipe Rojas -- 0542569\\Lehal Sherik -- 0543118}
\date{27 October 2017}

\maketitle

\section{Project subject}

Our OWL have seven classes, Category, Ebook, Purchase, Publisher, Author, User, Publisher and Role.

The \emph{User} class represent both an admin or a customer of the platfrom, it have seven attributes:

\begin{itemize}
  \item userId
  \item information
  \item name
  \item email
  \item password
  \item lastName
\end{itemize}

A \emph{User} of the system can either have the \emph{AdminRole} or just have the \emph{ClientRole},
relation like \emph{hasAdminRole} or \emph{hasClientRole} from the \emph{User} domain in the
\emph{AdminRole}/\emph{ClientRole} range.
Both \emph{AdminRole}/\emph{ClientRole} are subclasses of \emph{Role} they are used to be more specific
what permission a \emph{User} have but are characterized by the attributes:

\begin{itemize}
  \item roleId
  \item description
\end{itemize}

Having one of these role imply the \emph{User} can interract with the \emph{Ebook} has shown with
the relations \emph{hasPurchased} and \emph{manage}. \emph{Ebook} are the main component of the platform
because it's simply what we sell on it. They are defined by:

\begin{itemize}
  \item ISBN
  \item title
  \item year
  \item version
\end{itemize}

But also a remote \emhpt{Author} class bound to \emph{Ebook} by the \emph{hasWritten} class who have got the
following attributes:

\begin{itemize}
  \item authorId
  \item firstName
  \item lastName
\end{itemize}

Also defined with the \emph{Publisher} class bound from \emph{Ebook} by the relation \emph{isPublishedBy},
it is made of:

\begin{itemize}
  \item publisherId
  \item name
\end{itemize}

And finally \emph{Categorie} class bound by \emph{Ebook} by the \emph{hasCategory} relation, categories are just
a way to do search on ebooks by genre so the \emph{Category} class is defined by:

\begin{itemize}
  \item categoryId
  \item description
\end{itemize}

To enable customers to read ebooks they must first create \emph{Purchase} as you can see in our ontology a \emph{User}
make by the \emph{isMaking} relation \emph{Purchase}. \emph{Purchase} are a way to modelize ebooks the user paid to have
for each \emph{Purchase} the user have to make a transaction and they can contain more than one \emph{Ebook} has we can
see with the \emph{isPartOf} relation.

Other privilege a customer have is to rate the \emph{Ebook} he bought modelized by the \emph{Rating} class.
\emph{User} and \emph{Rating} are bound together by the \emph{hasRated} relation and make \emph{Ebook} in relation
with it with \emph{hasRating}. The attributes for \emph{Rating} are:

\begin{itemize}
  \item ratingId
  \item number
\end{itemize}

\section{Rules description}

First rule, if the category of a book is also a subCategory of another category, it implies that the book belongs to both category and subcategory.\\
Ebook(?x), Category(?y), Category(?z), subCategoryOf(?y, ?z), hasCategory(?x, ?y), differentFrom(?y, ?z) $\rightarrow$ hasCategory(?x, ?z)

Second rule, if an user has Purchased an ebook, and ebook is part of a purchase, then we infer that a user “HasPurchased” an ebook. \\
User(?x), Purchase(?y), Ebook(?z), isMaking(?x, ?y), isPartOf(?z, ?y) $\rightarrow$ hasPurchased(?x, ?z)

Third rule, if a user has rated an ebook and an ebook has been rating, infers that user purchased the book.\\
User(?x), Rating(?y), Ebook(?z), hasRating(?x, ?y), hasRating(?z, ?y) $\rightarrow$ hasPurchased(?x, ?z)

This last rule allows to infer that if an user has the role admin, that user can manage the ebooks on the system\\
User(?admin), Ebook(?book), AdminRole(?role), hasAdminRole(?admin, ?role) $\rightarrow$ manage(?admin, ?book)

\section{ER Schema}

\begin{center}
  \makebox[\textwidth]{\includegraphics[width=\paperwidth]{ER_graph.png}}
\end{center}

\section{WebOwl visualisation}
\begin{center}
  \makebox[\textwidth]{\includegraphics[width=\paperwidth]{webOwl.png}}
\end{center}


\end{document}
